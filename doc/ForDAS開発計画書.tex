\documentclass[10pt,a4paper]{jsarticle}

%- Figure/Table Setting
\usepackage[dvipdfmx]{graphicx,xcolor}
\graphicspath{./Figs/}
\usepackage{float}

%- Math Setting
\usepackage{amsmath,amssymb}

%- Font Setting
\usepackage[T1]{fontenc}
\usepackage{textcomp}
\usepackage[sc]{mathpazo}
\usepackage[scaled]{helvet}
\renewcommand{\ttdefault}{lmtt}
\usepackage{otf}

\renewcommand{\figurename}{Fig.\,\,}
\renewcommand{\tablename}{Table\,\,\,}

\begin{document} % 文書本体の開始
	
	\title{開発計画書}
	\author{田中 進夢}
	\maketitle
	
	
	%- Main Contents
	\section{開発物}
	簡易飛翔シミュレーションソフトウェア(機能追加)
	
	\section{開発責任者}
	情報処理班 田中進夢
	
	\section{依頼者}
	燃焼班 濃沼悠斗
	
	\section{開発内容}
	簡易シミュレーションによるモータ仕様探索機能の開発
	
	\section{開発期間}
	2017/10/09~2018/04/01(開発許容期間であり、作成に必要な期間は問わない)
	
	\section{設計方針}
	C\#を用いて開発する。既存である設計支援用簡易シミュ ForDAS に追加機能という形で実装する。入力と
	して構造質量、目標高度とその上限下限をとり、ツール内で酸化剤質量流量と燃焼時間を変数として探索解析
	を行い全力積を出力する。Isp、O/F などはアドバンスドオプションとして入力を用意する。
	
	
	
	
	\newpage
\end{document} % 文書終了